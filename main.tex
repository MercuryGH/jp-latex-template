% XeLaTeX 或 LuaLaTeX 编译
\documentclass{ctexart}
\usepackage{ruby}

\usepackage[a4paper, margin=2cm]{geometry}

% \renewcommand{\rubysep}{-4ex}
\renewcommand{\rubysep}{-0.4em}
\renewcommand{\rubysize}{0.5} % default: 0.4

\newCJKfontfamily\gothic{IPAexGothic}
\newCJKfontfamily\mincho{IPAexMincho}

\newcommand{\bbigbreak}{\bigbreak \bigbreak}

% \setCJKmainfont{Hiragino Mincho Pro}

\begin{document}

\section*{\gothic 森光子さんは自嘲気味に詠む}

\begin{center}
    \mincho
    天声人語

    2009年5月10日
\end{center}


\mincho
\ruby{絵}{え}に\ruby{描}{えが}いたような\ruby{美}{び}\ruby{形}{けい}ではないにしても、\ruby{独}{ひと}り\ruby{磨}{みが}いた演技には自信があった。下積み時代、\ruby{森}{もり}\ruby{光子}{みつこ}さんは\ruby{自}{じ}\ruby{嘲}{ちょう}\ruby{気}{ぎ}\ruby{味}{み}に\ruby{詠}{よ}む。〈あいつより うまいはずだが なぜ売れぬ〉。遅く咲かせた\ruby{償}{つぐな}いだろうか、芸の神様は果てしない開花を用意していた。

\begin{itemize}
    \mincho
    \item 森光子(1920 - 2012)、日本の女優、京都市出身。森さんは「放浪記」で上演2000回を達成し、国民栄誉賞を受けるまでに上り詰めた。文化勲章、紫綬褒章も受章。昭和と平成を代表する国民的女優として、テレビ、舞台、映画、司会、CMほか数々の作品を残している。テレビドラマでは「時間ですよ」で松の湯の女将(おかみ)を演じ、庶民的キャラクターが人気を呼んだ。ジャニーズたちのステージが大好きだった。「放浪記」の舞台では87歳まででんぐり返しを披露し続けた。15歳でデビュー、41歳から花開いた女優人生だったが、2012年11月10日、92歳で幕を下ろした。
    \rmfamily
    \begin{itemize}
        \item \mincho でんぐり返し:\rmfamily 翻跟斗。
    \end{itemize}
    \item \mincho あいつより \ うまいはずだが \ なぜ売れぬ:\rmfamily 这首五七五由森光子创作于在演艺界屈于人下的底层时期(下積み時代),也就是标题中的“有自嘲意义”的诗,表达了演技虽好但却不被重用的屈才之情。
\end{itemize}

\rmfamily
虽然没有照片上那么漂亮,但她对自己磨练出来的演技有信心。在演艺界的底层时期,森光子写了一首带有自嘲意味的诗: “和她比起来,明明演得更精彩,为何不用我?” 或许是要补偿她的大器晚成,艺术之神准备让她的演艺生涯无尽地绽放下去。

\bbigbreak

\mincho
森さんの舞台「放浪記」が上演2千回を達成した。単独主演の記録を更新中だ。初演は\ruby{ケネディ}{Kennedy}大統領が就任した1961(昭和36)年。先の大戦の前線を歌手として\ruby{慰}{い}\ruby{問}{もん}した女性が、89歳の誕生日を仕事場で迎えたことにも驚く。

\begin{itemize}
    \rmfamily
    \item 在表示“世界纪录”等的“纪录”时,日语不使用“纪录”,用“記録”代替。
    \mincho
    \item 1941年(昭和16年)、森光子は21歳で歌手を目指して陸軍の満洲慰問団に参加した。戦時中は日本軍慰問団で東海林太郎らの前座歌手としてミスワカナ・玉松一郎らと中国戦線や南方戦線を巡回する。
\end{itemize}

\rmfamily
森主演的舞台剧《放浪記》已上演了两千场,不断打破着单独出演主角次数的最高纪录。森第一次主演《放浪記》是在肯尼迪总统就职的1961年(昭和36年)。令人惊讶的是,这样一名于二战前线担任过歌手慰问士兵的女性,其 89 岁生日是在工作场所迎来的。

\bbigbreak

\mincho
大阪の舞台に出ていた森さんを、出張中の劇作家、\ruby{菊田}{きくた}\ruby{一夫}{かずお}が見つけたのは偶然だった。劇場での打ち合わせを終え、客席の後ろで空港への\ruby{ハイヤー}{hire}を待つ「伝説」の3分間。女優の\ruby{技}{わざ}を知るには十分だろう。

\rmfamily
出差的剧作家菊田一夫对正在大阪舞台上表演的森的发掘是一次偶然。当时,菊田结束了在剧院的商讨,正在观众席后面等待出租车前往机场。这“传奇”般的三分钟,也许足够知晓一个女演员的技艺了。

\bbigbreak

\mincho
ところが、東京に呼んでおいて「君は面白いが、やっぱりワキだな」である。\ruby{脇}{わき}\ruby{役}{やく}で行けとの助言に森さんは奮起し、菊田脚本・演出による放浪記の主人公、\ruby{林}{はやし}\ruby{芙美子}{ふみこ}役をつかむ。時に41歳、東京では駆け出し。17歳下のひばり、15歳下の\ruby{裕次郎}{ゆうじろう}ともすでに国民的スターだった。

\rmfamily
然而,当森给菊田打电话到东京时,他说:“你很有趣,但果然感觉你还是演配角吧。” 在被建议扮演配角后,森奋发图强,最终获得了菊田编剧和导演的《放浪記》中的主角林芙美子的角色。当时她41岁,刚刚在东京崭露头角。比他小17岁的ひばり(美空云雀,みそら \ ひばり)和比他小15岁的(石原)裕次郎都已经是国民明星了。

\bbigbreak

\mincho
2千回に同じものはない。共演の違いだけでなく、せりふ回しからたばこの吸い方まで、主役も舞台も進化してきた。長いファンは、新しい芙美子と元気な森さんの二人に会いに来る。その期待が花を終わらせない。

\rmfamily
《放浪記》的出演,两千次都各不相同。不仅共同出演(的演员)存在差异,而且从台词技巧到抽烟方式,主演和舞台也逐渐进化。老粉丝们纷纷前来会面新的芙美子和活泼的森,这些期望让作为主演的森永不凋零。

\bbigbreak

\mincho
生来の役者というのだろう。お酒は苦手なのに\ruby{粋}{いき}に\ruby{酔}{よ}い、\ruby{京}{きょう}\ruby{女}{おんな}ながら東京下町の香りをまとい、子どもを持たぬまま「日本のお母さん」になった。いま「\ruby{老}{お}い」さえもたぶらかし、新たな顔で記録を重ねていく。

\begin{itemize}
    \mincho
    \item 纏い(まとい)
    \item 誑かす(たぶらかす)
\end{itemize}

\rmfamily
大概这就是所谓的天生的演员吧。不擅长喝酒,却醉得潇洒;身为京都女人,却萦绕着老东京的香气;没有孩子,却成为了国民共同的母亲。如今,她连“老年”都欺骗了,以新姿态不断刷新着纪录。

\bbigbreak

% \section{汉字}

% \rmfamily
% 汉字,在中国亦称中文字、国字、唐字、方块字,是汉字文化圈广泛使用的一种文字,是

\end{document}